
\section{Management Plan}


% \emph{Provide a summary of expertise of the proposed team, including any subcontractors/consultants and key personnel who will be executing the work. Identify a principal investigator (PI) for the project. Provide a clear description of the team’s organization including an organization chart that includes, as applicable, the relationship of team members; unique capabilities of team members; task responsibilities of team members; teaming strategy among the team members; and key personnel with the amount of effort to be expended by each person during the project. Provide a detailed plan for coordination including explicit guidelines for interaction among collaborators/ subcontractors of the proposed project. Provide a plan for interacting with other SIMPLEX performers to the extent described in Section I. Include risk management approaches. Describe any formal teaming agreements that are required to execute this project. List Government-furnished materials or data assumed to be available.}

% SCHEMATIC
% 1) unique capabilities
% 2) task responsibilities
% 3) % effort



\begin{figure}[h!]
\centering
\includegraphics[width=1.0\linewidth]{../JHU-Proposal/JHU-Figs/JHU-SIMPLEX-management.pdf}
\caption[Orgchart.]{From RAGs to Riches Org-Chart.}
\label{fig:orgchat}
\end{figure}


PI Joshua Vogelstein has assembled a vertically intergrated team of individuals with complementary expertise and a long successfully history of working well together.  Importantly, PI Vogelstein, co-I Priebe, and key personnel, although in different departments with different backgrounds, all work on the same floor in the same building at Johns Hopkins University.  Co-I Burns's office is just across the quad.  The PI and the two co-I's share graduate students and post-docs, in fact, some of PI Vogelstein's students work in co-I Burns's lab space.  The three investigators are jointly funded through both XDATA and GRAPHS; separately, co-I Burns and PI Vogelstein are jointly funded on several NIH awards, as well as private funding.  

The RAGs to Riches team, as illustrated in Figure \ref{fig:orgchat}, is tightly integrated.  Each member of the team will be responsible for a single task throughout all three phases of the program:
\begin{compactenum}
\item Co-I Priebe will manage Task 1, all aspects of mathematical framework, working closely with co-I Burns to ensure scalability of the developed methodology. Priebe will also work with PI Vogelstein, and oversee both of the key personnel, Nam Lee and Youngser Park.
\item Co-I Burns will manage all Task 2 activities, including design and administration of computational resources, and implementation of all scalable algorithms, databases, and interfaces.
\item PI Vogelstein will manage Task 3, datafication, extensive utilizing the ideas of co-I Priebe and implementations of co-I Burns.  
\item Key Personnel Nam Lee will be responsible for Task 4, discovery.  This will largely entail taking the registered data products from PI Vogelstein, and deploying the methodologies developed by co-I's Priebe and Burns.
\item Key Personnel Youngser Park will contribute to all Task 1 efforts with co-I Priebe, as well a work closely on discovery with Dr. Lee.  Dr. Park will also assist Dr. Vogelstein with project management duties and his task responsibilities will include assisting with the coordination of meetings, technical reports.  He has served as Project Manager for Dr. Carey Priebe in various projects in the past.
\end{compactenum}

The team will collectively oversee three graduate students for the duration of the program.  The three investigators will each be responsible for the activity of one of the graduate students.  The close-knitness of the team will mean that all three graduate students have extensive interactions with each of the team members.

Dr. Vogelstein will serve as the main point of contact with the DARPA PM.  For Task 5, he will  1) oversee all of the internal JHU efforts for the proposed project, 2) manage the budget, 3) oversee writing of reports and research papers, and maintenance of project websites, 4) oversee participation in face-to-face meetings involving travel.  The PI will also manage Task 3, datafication, extensively utilizing the ideas of co-I Priebe and implementations of co-I Burns.

The team will communicate through regular weekly meetings, which will be conducted in person or by teleconference and include all team members. Overall and individual task progress will be assessed monthly and recommendations for any necessary effort focus or modifications to meet program milestones will be issued by the PI. \textbf{Identifying risks due to technical issues and establishing mitigation strategies will be addressed at the weekly teleconference.} This will include mitigation of dependency risk, for example, FlashGraph implementations of context-aware algorithms must follow from the key development of the intial algorithms in Phase I. In the event of impending risk, PI Vogelstein will re-assign teammates appropriately to ensure successful completion of all milestones in the proposed timeline.  The PI and other project staff as appropriate will participate in all DARPA meetings and collaboration meetings with TA1 partners, as determined by DARPA.  Reports and software will be shared between all team members though websites hosted at JHU. Publication of all research work will be encouraged.

Importantly, the groups that are contributing to the data that the team will use for datafication and discovery are all in regular contact with the team.  Twice annually, members of the team will meet members from the data contributors laboratory for extensive face-to-face time.  Already the team has a day long scheduled meeting with Karl Deisseroth's data collection guru.  

Finally, we note that this team already publishes together extensively in a wide range of venues, including the top computer science, applied mathematics, statistics, and neuroscience journals.  Moreover, we also have already co-published with the data generator groups, and have several additional publications at various stages of acceptance.  PI Vogelstein will ensure that everything runs smoothly, and will intervene, as necessary, to guarantee successful program completion.


% \mytodo{``Include risk management approaches.'' What does this mean???}

