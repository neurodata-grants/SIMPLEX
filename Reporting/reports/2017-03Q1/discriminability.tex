\documentclass[simplex.tex]{subfiles}
% NO NEED TO INPUT PREAMBLES HERE
% packages are inherited; you can compile this on its own
\providecommand{\mb}[1]{\boldsymbol{#1}}
\providecommand{\mv}[1]{\vec{#1}}
\providecommand{\ve}[1]{\boldsymbol{#1}}
\newcommand{\bpsi}{\ve{\psi}}
\newcommand{\bv}{\mb{v}}
\newcommand{\bx}{\mb{x}}
\newcommand{\bA}{\mathbf{A}}
\newcommand{\bH}{\mathbf{H}}
\newcommand{\bX}{\mathbf{X}}
\newcommand{\bP}{\mathbf{P}}
\newcommand{\bB}{\mathbf{B}}
\newcommand{\bD}{\mathbf{D}}
\newcommand{\bS}{\mathbf{S}}
\newcommand{\bR}{\mathbf{R}}
\newcommand{\bJ}{\mathbf{J}}
\newcommand{\bLambda}{\mathbf{\Lambda}}
\begin{document}
\subsection{Discriminability}
We develop a measure of discriminability (or reliability).  It
is intuitive to understand and easy to implement.
Discriminability is defined to be the probability of within
subject distances being smaller than the cross subject
distances. If we let $x_{i,t}$ denote the t$^{\text{th}}$ trial of subject 
$i$ and $\Delta(,)$ be the metric, the (population) discriminability $D$
is: 
\[D:= P (\Delta(x_{i,t} , x_{i,t’}) \leq  \Delta(x_{i,t} , x_{i’,t’’})).\]
Previously, we search for the optimal processing pipeline which has the
maximal discriminability. 
\\
Currently, we are considering the discriminability from a different perspective. Specifically, we want to use the discriminability as an internal measure of the consistency in clustering. If we let $x_{i,t}$ denote the t$^{\text{th}}$ sample of cluster 
$i$ and $\Delta(,)$ be the metric, the discriminability $D$
is: $D:= P (\Delta(x_{i,t} , x_{i,t’}) \leq  \Delta(x_{i,t} , x_{i’,t’’}))$. Large discriminability implies the clusters are more consistent, that is we can better differentiate samples from different clusters. 
\\
We are also developing a clustering algorithm \ref{alg:dcl} which maximizes the discriminability. The algorithm will assign each sample $i$ a cluster identity $k_i$ such that the discriminability is maximized. The algorithm is similar to K-means, but we believe discriminability provides a more robust measure than within cluster distances which is maximized in K-means.
\begin{algorithm}               
	\caption{Cluster samples through maximizing discriminability.  }   
	\label{alg:dcl}                       
	\begin{algorithmic}                    
		\Require Samples $\{\bx_{i}\}$ and number of clusters $K$.
		\Ensure Cluster identity $\{k_{i}\}$. 
		\Function{DiscriminabilityClustering}{}
		\State Initialize $\{k_{i}\}$ randomly
		\While{ not convergent}
		\For{$i$} 
		\For{$j$ in $1:K$}
		\State Set $k_{i} = j$ 
		\State ComputeDiscriminability($\{\bx_{i}\}$,$\{k_{i}\}$)
		\EndFor
		\State Set $k_{i}$ to the cluster with maximum discriminability
		\EndFor 
		\EndWhile
		\State Output $\{k_{i}\}$ 
		\EndFunction
	\end{algorithmic}
\end{algorithm}


\end{document}
