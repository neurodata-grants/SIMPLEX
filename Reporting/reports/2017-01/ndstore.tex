\documentclass[simplex.tex]{subfiles}
% NO NEED TO INPUT PREAMBLES HERE
% packages are inherited from simplex.tex; you can compile this on its own

\onlyinsubfile{
\title{NeuroData SIMPLEX Report: Subfile}
}

\begin{document}
\onlyinsubfile{
\maketitle
\thispagestyle{empty}

The following report documents the progress made by the labs of Randal~Burns and Joshua~T.~Vogelstein at Johns Hopkins University towards goals set by the DARPA SIMPLEX grant.

%%%% Table of Contents
\tableofcontents

%%%% Publications
\bibliographystyle{IEEEtran}
\begin{spacing}{0.5}
\section*{Publications, Presentations, and Talks}
\vspace{-20pt}
\nocite{*}
{\footnotesize	\bibliography{simplex}}
\end{spacing}
%%%% End Publications
}

\subsection{ndstore}

We continue to now migrate all our annotation datasets to the cloud. The annotation projects will now be hosted in MySQL backed with AWS EBS storage and will be converted to AWS DynamoDB soon. There will be more than 30 odd annotation datasets avaliable in the cloud for public use. The datasets can be accessed at \href{http://neurodata.io}.

We added support for new resource and authentication web-servics in our python wrapper called ndio. This will allow our users to continue to use ndio for future interactions with the web back-end. They can now utilize the new web-services we have added and create resources using this which is more easier then using the RESTful web-services directly.

We are also in the process of deploying a status page for our web-services. The status page will act as a dashboard for all our users and a single point of contact for them to check if our services are online or suffering from an outage. It will also allow us to inform all our users who are subsribed to the status page of future planned outages. This will streamline our services and is standard industry practice for other companies running web-services. The status page is located here \href{https://neurodata.statuspage.io/}.

The MRI ingest service is now active and being used by some users in the lab to ingest data into ndstore. All of this service was already deployed in the cloud and the ingested data will be avaliable at \href{http://mri.neurodata.io}.


\subsubsection{ndingest}

The access policies for the ndingest is now complete. We are currrently testing our the new service and will soon release it in beta mode to some our close collaborators. This service will allow us to speedup our data ingest rates manifold. We plan to benchmark this service once we are done deploying it.

The ingest client developed witj JHU-APL, has now been converted so that it can use multiple threads in python. This capability will allow us to upload multiple slices of the data simultaneously and allow us to upload data to the cloud at a much faster rate. 

\end{document}
