\documentclass[simplex.tex]{subfiles}
% DO NOT INCLUDE PREAMBLES/PACKAGES HERE!!
% packages are inherited from preamble.tex; you can compile this on its own
\begin{document}
\subsection[ndstore]{ndstore}
%% April Boss Update

With the release of Python 3.6 in December of last year, Python 3 has now reached sufficient maturity for us to begin migrating our infrastructure from Python 2.7 to Python 3.x. In addition to the large number of new features in the programming language that were introduced in Python 3, significant performance and stability improvements have also been introduced. We are particularly interested in using the much richer set of exceptions and increased performance for iterators present in Python 3. As a user-facing web service, NDStore must be both performant and stable, and return meaningful error messages to users when something goes wrong (e.g. bad user parameters, an internal system issue, or even a bug in the code). 


To that end, we have been working with our collaborators at JHUAPL to move to a standardized Python 3 codebase for large (teravoxel) imaging volumes. We have successfully deployed a Python 3 endpoint, caching infrastructure, and some miscellaneous background processing modules (e.g for data ingest) in Amazon Web Services. We are now working to determine how best to unify our existing infrastructure with our new Python 3 infrastructure, with the overarching goal of maintaining a consistent interface across all of our assorted data types (e.g. electron and light microscopy, functional time-series imaging, magnetic resonance imaging) while minimizing duplication of effort between JHU and JHUAPL. 


We have also spent considerable effort testing the data ingest client among JHU team members. We are hoping to deploy the ingest client to collaborators in the coming months. Allowing collaborators to ingest data with minimal JHU intervention increases the corpus of data we are able to collect, which provides more resources for training, running algorithms, or doing analysis. 

\clearpage
\end{document}
