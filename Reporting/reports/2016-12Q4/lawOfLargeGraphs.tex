\documentclass[simplex.tex]{subfiles}
% NO NEED TO INPUT PREAMBLES HERE
% packages are inherited; you can compile this on its own

\onlyinsubfile{
\title{NeuroData SIMPLEX Report: Subfile}
}

\begin{document}
\onlyinsubfile{
\maketitle
\thispagestyle{empty}

The following report documents the progress made by the labs of Randal~Burns and Joshua~T.~Vogelstein at Johns Hopkins University towards goals set by the DARPA SIMPLEX grant.

%%%% Table of Contents
\tableofcontents

%%%% Publications
\bibliographystyle{IEEEtran}
\begin{spacing}{0.5}
\section*{Publications, Presentations, and Talks}
%\vspace{-20pt}
\nocite{*}
{\footnotesize	\bibliography{simplex}}
\end{spacing}
%%%% End Publications
}

\subsection{Law of Large Graphs}

We have submitted it to PLOS Computational Biology. And now here
on the arXiv is our first step:
\href{http://arxiv.org/abs/1609.01672}{http://arxiv.org/abs/1609.01672}

Estimating the mean of a population of graphs based on a sample
is a core problem in network science. Often, this problem is
especially difficult because the sample or cohort size is
relatively small as compared to the number of parameters to
estimate. While using the element-wise sample mean of the
adjacency matrices is a common approach, this method does not
exploit any underlying graph structure. We propose using a
low-rank method together with tools for dimension selection and
diagonal augmentation to improve performance over the naive
methodology for small sample sizes. Theoretical results for the
stochastic blockmodel show that this method will offer major
improvements when there are many vertices. Similarly, in
analyzing human connectome data, we demonstrate that the low-rank
methods outperform the standard sample mean for many settings.
These results indicate that low-rank methods should be a key part
of the tool box for researchers studying populations of graphs.


\end{document}
